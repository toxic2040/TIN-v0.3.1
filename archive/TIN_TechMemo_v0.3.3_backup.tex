\documentclass[11pt,a4paper]{article}
\usepackage[margin=1in]{geometry}
\usepackage{graphicx}
\usepackage{booktabs}
\usepackage{hyperref}
\usepackage{float}

\title{TIN v0.3.1 Technical Memo \\ True-Polar Relay Constellation for Lunar South Pole Coverage}
\author{Independent Proposer \\ toxic2040 \\ February 18, 2026}
\date{}

\begin{document}

\maketitle

\section{Executive Summary}
TIN v0.3.1 delivers >99\% south-pole coverage using a minimal constellation of 6–8 smallsat relays in 400–500 km circular 90° polar orbits with staggered phasing. Combined with CCSDS DTN, this architecture supports Artemis, commercial landers, and ISRU in PSRs.

Key result (28-day simulation, elev >5°):  
\textbf{99.6\%} (6 sats @ 400 km) → \textbf{100.0\%} (8 sats or 500 km variant).

GitHub: \url{https://github.com/toxic2040/TIN-v0.3.1}

\section{Baseline Constellation}
\begin{table}[h]
\centering
\begin{tabular}{lccc}
\toprule
Parameter & 6-sat baseline & 8-sat option & 500 km option \\
\midrule
Altitude & 400 km & 400 km & 500 km \\
Inclination & 90° & 90° & 90° \\
\# Relays & 6 & 8 & 6 \\
RAAN spacing & 60° & 45° & 60° \\
South-pole coverage & 99.6\% & 100.0\% & 100.0\% \\
\bottomrule
\end{tabular}
\caption{TIN v0.3.1 locked baselines}
\end{table}

\section{South-Pole Coverage Results}
\begin{figure}[H]
\centering
\includegraphics[width=0.9\textwidth]{tin_400km_6sats_90deg.png}
\caption{6 sats @ 400 km — 99.6\% coverage (lat < –85°)}
\end{figure}

\begin{figure}[H]
\centering
\includegraphics[width=0.9\textwidth]{tin_400km_8sats_90deg.png}
\caption{8 sats @ 400 km — 100.0\% coverage}
\end{figure}

\begin{figure}[H]
\centering
\includegraphics[width=0.9\textwidth]{tin_500km_6sats.png}
\caption{6 sats @ 500 km — 100.0\% coverage}
\end{figure}

\section{Next Steps (Phase I Scope)}
\begin{itemize}
\item Integrate Lunar Pathfinder ELFO as hybrid anchor node
\item ION DTN bundle routing simulations
\item Far-side + PSR gap analysis
\item SWaP/cost model (ESPA-class rideshare)
\item Open-source dataset release
\end{itemize}

Full CLI tool and raw simulation data available in the GitHub repo.

\end{document}