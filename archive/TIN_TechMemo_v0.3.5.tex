
\documentclass[11pt,a4paper]{article}
\usepackage[utf8]{inputenc}
\usepackage[T1]{fontenc}
\usepackage{graphicx}
\usepackage{booktabs}
\usepackage{geometry}
\geometry{margin=1in}

\title{TIN v0.3.5 Technical Memo \\
Hybrid Polar Relay Constellation + Lunar Pathfinder ELFO Hub \\
for Artemis South Pole \& Far-Side Coverage}

\author{Independent Proposer \\
toxic2040}

\date{February 19, 2026}

\begin{document}

\maketitle

\section{Executive Summary}

TIN v0.3.5 delivers near-perfect south-pole coverage (99.9–100.0 \%) and major far-side boost (63.2–68.5 \%) using a hybrid architecture: 6–8 smallsat relays in 400 km circular 90° polar orbits + the real Lunar Pathfinder ELFO (a=5740 km, e=0.58, i=55°, frozen arg peri ~86°, perilune over south pole) as primary intelligent DTN/AI routing hub. Combined with CCSDS DTN, this supports Artemis, commercial landers, PSR ISRU, and far-side operations.

Key results (28-day simulation, elev >5°):

\begin{table}[h]
\centering
\begin{tabular}{lcc}
\toprule
Configuration & South Pole (\%) & Far-Side (\%) \\
\midrule
Pure Polar 6 sats @ 400 km & 99.6 & 46.4 \\
Pure Polar 8 sats @ 400 km & 100.0 & 54.4 \\
\textbf{Hybrid 6 polar + Pathfinder} & \textbf{99.9} & \textbf{63.2} \\
\textbf{Hybrid 8 polar + Pathfinder} & \textbf{100.0} & \textbf{68.5} \\
\bottomrule
\end{tabular}
\caption{v0.3.5 Hybrid results}
\end{table}

GitHub: https://github.com/toxic2040/TIN-v0.3.1

\section{Baseline Constellation}

\begin{table}[h]
\centering
\begin{tabular}{lccc}
\toprule
Parameter & 6-sat baseline & 8-sat option & 500 km option \\
\midrule
Altitude & 400 km & 400 km & 500 km \\
Inclination & 90° & 90° & 90° \\
\# Relays & 6 & 8 & 6 \\
RAAN spacing & 60° & 45° & 60° \\
South-pole coverage & 99.6\% & 100.0\% & 100.0\% \\
\bottomrule
\end{tabular}
\caption{TIN v0.3.1 locked baselines (pure polar)}
\end{table}

\section{Hybrid Coverage Results (v0.3.5)}

\begin{figure}[h]
\centering
\includegraphics[width=\textwidth]{tin_hybrid_6polar_ELFO_south.png}
\caption{Hybrid 6 Polar + ELFO — South Pole Coverage (99.9 \%)}
\end{figure}

\begin{figure}[h]
\centering
\includegraphics[width=\textwidth]{tin_hybrid_8polar_ELFO_south.png}
\caption{Hybrid 8 Polar + ELFO — South Pole Coverage (100.0 \%)}
\end{figure}

\begin{figure}[h]
\centering
\includegraphics[width=\textwidth]{tin_hybrid_6polar_ELFO_far.png}
\caption{Hybrid 6 Polar + ELFO — Far-Side Coverage (63.2 \%)}
\end{figure}

\begin{figure}[h]
\centering
\includegraphics[width=\textwidth]{tin_hybrid_8polar_ELFO_far.png}
\caption{Hybrid 8 Polar + ELFO — Far-Side Coverage (68.5 \%)}
\end{figure}

The ELFO perilune passes provide the critical far-side visibility boost while the polar constellation guarantees near-perfect south-pole service. This hybrid polar + frozen-elliptical architecture is deliberately generalizable to Mars, Venus, outer-planet moons, and solar-polar networks.

\section{Next Steps (Phase I Scope)}

\begin{itemize}
\item Integrate Lunar Pathfinder ELFO as hybrid anchor node (completed in v0.3.5)
\item ION DTN bundle routing simulations
\item Far-side + PSR gap analysis
\item SWaP/cost model (ESPA-class rideshare)
\item Open-source dataset release
\end{itemize}

Full CLI tool and raw simulation data available in the GitHub repo.

\end{document}
